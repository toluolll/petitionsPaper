%!TEX root = ../sig-alternate.tex
\section{Introduction}
\label{sec:intro}

% Context: 
Environment and wildlife issues are linked to the human-induced climate change[REFERENCE] and therefore, it is important to analyze people's engagement into its resolution.
%
Changes in environment affect humanity not only online but rather offline. Therefore, public campaigns is one of the major instruments to convey various information from communities to the mass, and backward, from people to the stakeholders.
%
Since Twitter remains one of the easiest way to disseminate information on the Web to a large groups of people, we analyse the major instrument of public campaigning - petitions - on Twitter in case of climate change and animal welfare campaigns.

% Existing opportunity
%
Outcome of the petition is relatively easy to extract, since such information is usually unleashed by petition aggregator web cites and thus can be used to profile performance of the public campaigns through petition success.
% 1.
As petitions are the most visible, it is important for campaign leaders to know how they used by other campaigns and if a petitions success is caused by any of the Twitter platform aspects (users that tweet the petitions, petitions tweets outside of the campaign, type of campaign, final goal of the petitions etc.)
% 2. 
The technique could be useful as well for the across particular corpus dependencies.

% Our goal
%   Research questions and major findings:
In this work we tackle two main research questions: 
\textit{What is the role of petitions in various types of environmental campaigns?} We perform a comparative analysis of the environmental campaigns performed during 2009-2015 by their usage of petition as one of the actions and observe mobilization campaigns as the main adopters of this instrument.
\textit{What makes a petition on Twitter attractive in the end?} In this piece of word we answer this question by making an analysis of the petition features both in and out of the campaign frame. We propose a set of representative social, content and campaign features and reveal the major factors affecting the petition success. Additionally, we release an annotated corpus with the petitions, their tweets and outcomes\footnote{Link is removed to ensure anonymity.}. 

% Related work and our position/contributions x 2 paragraphs
% Campaigns/Protests on Climate Change
Climate change is highly discussed topic by various segments of society. \cite{Kirilenko2014} overviews the climate change domain, its polarization, discussion over time etc.
Variety of non-governmental campaigns advertise for increasing awareness and mobilizing people.
\cite{Tufekci2013} describes how on-line attention can driven towards particular politicized persona, while \cite{gonzalez2013networked} analyses information transmission during protests.
\cite{hestres2013preaching} studies public mobilization and online-to-offline social movement strategies for two major environmental movements. We analyze a great variety of environmental campaigns from the perspective of petition utilization.
% TODO: Possible add to the previous section one of the findings we will have!!!

% Petitions
Various works were conducting on analysis of the e-petitions conducted on various petition aggregators.
\cite{Hale2013} describe a temporal analysis of 8K petition on the UK No. 10 Downing Street and make an observation towards early signs of successful petition (large number of signatures during the first days).
\cite{Huang2015} analyses power users that produce petitions and has shown that only 1\% of general topic petitions on \url{change.org} reaches their goal.
However, to the best of our knowledge, no one has studied success analysis of the environmental petition based on the internal and external attributes of the corresponding public campaign on Twitter.
% Kickstarter
On the other hand, analysis of the e-petitions can be compared to the crowdfunding, since in both fields desired and obtained support can be analysed. \cite{Etter2013} studies various prediction techniques of the Kickstarter campaigns.
Moreover, \cite{An2014} analyses investor activity on Kickstarter and make recommendations of projects based on their Twitter activity. Unlike aforementioned works, we focus on the climate change and animal welfare petitions, as a part of the environmental public campaigns on Twitter.


% Implications/Impact etc...
Only 25\% of the petition posted with hashtags of the environmental campaigns on Twitter received necessary number of signatures. It is important to know whether success of a particular petition correlates with a particular campaigns or distribution outside of the campaign on Twitter.
The techniques presented below are not restricted to the environmental domain and could be applied to any similar setting.

% Disclaimers
In Section~\ref{sec:dataset}, we describe our dataset, its main characteristics, filtering and annotation we apply. We then present in Section~\ref{sec:petition_analysis} petitions' position within environmental campaigns and outside of then within Twitter platform. Finally, we highlight major finding (Section~\ref{sec:discussion}) and conclude in the end.