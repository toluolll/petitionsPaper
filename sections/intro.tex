%!TEX root = ../sig-alternate.tex
\section{Introduction}
\label{sec:intro}

% Context: 
% Environment and wildlife issues are linked to the human-induced climate change~\cite{solomon2009irreversible} and therefore, humanity is concerned about the ways to improve the situation.
%
% As a result, public campaigns have became one of the major instruments to increase awareness and mobilize people towards issues of climate change and animal welfare~\cite{Pearce2014}.
%
% In turn, public campaigns often leverage e-petitioning~\cite{mosca2009petitioning} to reach out to local, national governments or institutions.
% Outcome of the petitions is relatively easy to extract, since such information is usually provided by petition aggregator web cites. This information can help quantify the performance of the public campaigns and petitions themselves, i.e., by a successful petition we understand a petition that has reached specified number of signatures.

The discourse on climate change is often focused on the impact it has on the environment and on  wildlife~\citeauthor{solomon2009irreversible} \shortcite{solomon2009irreversible}.
To bring those issues in the public spotlight, social media campaigns have proved to be an effective instrument to raise awareness and mobilize masses~\citeauthor{Pearce2014} \shortcite{solomon2009irreversible}.
To further push for concrete actions from governments or public entities, many campaigns resort to e-petitioning~\citeauthor{mosca2009petitioning} \shortcite{solomon2009irreversible}, whose success is also much easier to assess: reaching or not a required number of signatures.
Information about the number of signatures obtained for a given e-petition is often publicly available via e-petitions aggregators websites such as \url{thepetitionsite.com}, \url{avaaz.org}, \url{change.org} etc., and can be used as a proxy for the performance of the public campaigns and petitions themselves.

% Existing opportunity
%
% 1.
% E-petitions share many features and qualities with other social platforms, e.g., can create and share content.
% Since Twitter remains one of the easiest way to disseminate information on the Web to a large groups of people, we analyse tweets mentioning petitions posted with the environmental campaign hashtags on Twitter.
% It is important for the campaign leaders to know how petitions are promoted by other campaigns on Twitter and if a petition success is caused by any of the Twitter platform aspects (users that tweet the petitions, petitions tweets outside of the campaign, type of campaign, final goal of the petitions etc.)

% Our goal
%   Research questions and major findings:
In this work, we tackle two main research questions.

\textbf{RQ1:} \textit{Which types of the public campaigns use petitions in their agenda?}
% What is the role of petitions in various types of environmental campaigns?}
To answer this question, we study several environmental campaigns that were run in the beginning of 2015, measuring the incidence of e-petitioning as an instrument for promoting different types of campaigns (awareness, mobilization). We find that petitioning is particularly important for mobilization campaigns.\footnote{Mobilization campaigns refer to the campaigns whose primary goal is to engage and motivate a wide range of partners, allies and individual at the national and local levels towards a particular problem or issue, while awareness campaigns refer to the campaigns whose primary goal is to raise people’s awareness regarding a particular subject, issue, or situation.}

\textbf{RQ2:} \textit{What makes a petition promoted by a public campaign successful?} We answer this question by making a feature analysis and comparing tweets that belong to public campaigns to individual tweets. 
%
We propose a set of social and contextual features and show how the required number of signatures for an environmental petition is correlated to its outcome.
Additionally, we release an annotated corpus with the petitions, their corresponding tweets and outcomes\footnote{https://github.com/toluolll/PetitionsDataRelease}.
For this study we focus on Twitter, which remains one of the main channels for social media campaigns, also providing relatively easy access to campaign data.

% Related work and our position/contributions x 2 paragraphs
% Campaigns/Protests on Climate Change
\textbf{Climate Change Discourse on Social Media.} Climate change is a highly discussed topic.
\citeauthor{Kirilenko2014} \shortcite{Kirilenko2014} overview the climate change domain, its polarization, discussion over time etc.
\citeauthor{Olteanu2015} \shortcite{Olteanu2015} study how various climate-related events are highlighted by various media sources.
A variety of public campaigns use social platforms to increase awareness or mobilize people~\citeauthor{Mahmud2014} \shortcite{Mahmud2014}.
\citeauthor{Tufekci2013} \shortcite{Tufekci2013} describes how online attention can be driven towards particular politicized persona, while \citeauthor{gonzalez2013networked} \shortcite{gonzalez2013networked} analyzes information transmission during protests.
\citeauthor{hestres2013preaching} \shortcite{hestres2013preaching} studies public mobilization and online-to-offline social movement strategies for two major environmental movements.
Unlike this prior work, we analyze over 100 environmental campaigns as well as their effects on the success of petitions.
% TODO: Possible add to the previous section one of the findings we will have!!!

% Petitions
\textbf{Characterizing E-petitions.} Various studies were conducted to analyze e-petitions on various petition aggregators.
\citeauthor{Hale2013} \shortcite{hestres2013preaching} describe a temporal analysis of 8K petitions and discuss early signs of success (e.g., large number of signatures during the first days).
\citeauthor{Huang2015} \shortcite{Huang2015} analyze ``power'' users that produce petitions. The authors have shown that only 1\% of general petitions on \url{change.org} reaches their goal.
However, to the best of our knowledge, we are the first to analyze which factors predict the success of an environmental petition based on the internal and external attributes of the corresponding public campaign on Twitter.
% Kickstarter
On the other hand, e-petitions can be compared to crowdfunding, as both efforts work towards obtaining a given level of support over a short period of time.
\citeauthor{Etter2013} \shortcite{Etter2013} study various prediction techniques for Kickstarter campaigns.
Later, \citeauthor{An2014} \shortcite{An2014} analyze investor activity on Kickstarter and make recommendations based on their activity on Twitter. Unlike those works, we focus on environmental campaigns and petitions 
% and on environmental public campaigns 
on Twitter.


% Implications/Impact etc...
In this work, we found that 25\% of the petitions posted with environmental campaigns hashtags on Twitter obtained their required number of signatures.
Moreover, we identified a number of features that can act as indicators for the success of the petitions.
This information might be of interest to environmental activists and campaign leaders as it can influence the success of the message they are conveying to the public.
We also note that the techniques presented below are not restricted to the environmental domain and could be applied to any related setting.

% Disclaimers
% In Section~\ref{sec:dataset}, we describe our dataset, its main characteristics, filtering and annotation we apply. We then present in Section~\ref{sec:petition_analysis} petitions' position within environmental campaigns and outside of then within Twitter platform. Finally, we highlight major finding (Section~\ref{sec:discussion}) and conclude in the end.