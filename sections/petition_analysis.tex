%!TEX root = ../sig-alternate.tex
\section{Petition analysis}
\label{sec:petition_analysis}

Given a list of petition that gained tracking within climate change and animal welfare campaign on Twitter (described above), we, first, present an analysis on the petition position within different types of public campaigns and, second, analyse petition spread outside of the campaign. Finally, we propose a model to predict success of a particular campaign petition on Twitter.

\subsection{Petition in public campaigns on Twitter}
% Information regarding environmental campaigns, types etc.
% Our findings:
% (a) only mobilization and single awareness; annual campaign with big number of msgs and failure rate.
% (b) doimated by the ever-growing campaigns.
% (c) Interesting examples (http://www.petitions.moveon.org/sign/legalize-hemp-farming/ - askdrh;
% https://you.38degrees.org.uk/petitions/stop-the-fracking-cover-up-by-defra - talkfracking (fracking censorship to delete.))

Out of the 118 environmental campaigns in our dataset, only 20 campaign hashtags co-occur with a ``petition'' keyword.
Interestingly, the petitions were mainly initiated and promoted by mobilization campaigns, while only two awareness campaigns published petition with their hashtags, e.g., ``\#talkfracking'', ``\#worldlovefordolphins''. Interestingly, awareness petition were not directed towards a particular action but rather long term optimisations, e.g., preventing ``covering up'' hydraulic fracturing by some organizations, or legalize hemp farming.

From the perspective of user engagement pattern for the campaigns mentioning petitions over 60\% of them were identified as ever-growing, while the rest were distributed between ``annual'' and ``inactive'' campaigns. Not surprisingly, ``one-day'' campaigns do not tend to use petitions as their instruments since they use more short-term leverage. On the other hand, ``annual'' campaigns tend to have a large number of different petitions thus the fail rate is higher.

Surprisingly, while Kickstarter\footnote{kickstarter.com} campaigns~\cite{Etter2013} tend to fail for high final goals, while completely opposite situation can be observed for, usually non-profitable, environmental campaigns.

\subsection{Environmental petition on Twitter}
\jp{\lipsum[1-9]}

\subsection{Data analysis}
\label{sec:discussion}
\jp{\lipsum[1-2]}