%!TEX root = ../sig-alternate.tex
\section{Conclusions}
% Conclusions and main results.
In this paper we introduce a dataset of environmental petitions that were promoted by major environmental public campaigns on Twitter.
We study the petition role as one of the actions performed by a public campaign.
We propose a model to identify successful petitions by their presence on Twitter and highlight the main aspects featuring a petition to obtain required number of signatures.
% We show that the more critical petition is, the more likely it is to have a high signature goal and the more likely it is to collect signatures.
Although, our dataset is limited in size, we made a best effort data collection and cleaning, and could observe the spread of the petitions within the public environmental campaigns and identify the major factors that may lead to the success of the petition.
% TODO: Add materials about the petitions from the "thepetition.com"... if I have time to do it!!!!
Our findings can provide helpful directions for all leaders of the public campaigns, its participants, petition initiators and signers.

% Future work and limitation.
% 1.
Interesting future direction is to study the user aspect of the petition promoters on Twitter. In particular, we could identify the relations between petition signers and users who promote petitions on Twitter. The main difficulty here is to obtain this information for the large number of petition.

% 2.
Currently we propose a set of basic features and highlight the most valuable ones. The next step would be to explore time series properties of signatures, as well as, give actionable feedback on how to increase number of signers.