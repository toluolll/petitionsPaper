%!TEX root = ../sig-alternate.tex
\section{Conclusions}
% Conclusions and main results.
In this paper, we introduced a dataset of environmental petitions that were promoted by major environmental campaigns on Twitter.
We studied the use of petitions as one of the instruments of a public campaign.
We proposed a model to identify successful petitions and highlighted key aspects to obtain the required number of signatures.
% We show that the more critical petition is, the more likely it is to have a high signature goal and the more likely it is to collect signatures.
Although our dataset is limited in size, we could observe the petitions spread within the environmental campaigns and identify the major factors that lead to the success of the petitions.
Our findings provide helpful directions for all public campaigns, its participants, petition initiators, and signers.

% Future work and limitation.
% 1.
As future work, we plan to enhance the petition dataset by repeating the collection process over years.
Another interesting direction would be to study user dimensions of the petition promoters on Twitter. In particular, we would like to identify the relations between petition signers and users who promote petitions on Twitter. The main difficulty here is to obtain this information for a large number of petitions.

% 2.
In this piece of work, we quantified the positive effects of the intense petition promotion on Twitter, e.g., the number of retweets, unique users, user followers and attention uppercased words correlating to successful petitions.
The next step would be to explore the time series of the signatures, as well as to give actionable feedback on how to increase the number of signers over time.

\noindent \textbf{Acknowledgements.} The authors would like to thank Alexandra Olteanu for suggestions and feedback.