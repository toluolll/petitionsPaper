%\documentclass{acm_proc_article-sp}
\documentclass[letterpaper]{article}
\usepackage{aaai}

\usepackage{times}
\usepackage{helvet}
\usepackage{courier}
\usepackage{url}
\usepackage{dsfont}
\usepackage{subfigure}
\usepackage{graphicx}
\usepackage{xspace}
\usepackage{multirow}
\usepackage{booktabs}
\usepackage{enumitem}
\usepackage{listings}
\usepackage{siunitx}
\usepackage{balance}
\usepackage{lastpage}
\usepackage[usenames,svgnames]{xcolor}
\usepackage{amsmath,array,graphicx}
\usepackage{kantlipsum}
\usepackage{appendix}
\usepackage{lipsum}
\usepackage{sidecap}
\usepackage{sparklines}
\usepackage{pgfplots}
\usepackage{tikz}
\usepackage[font=small,labelfont=bf]{caption}
\newcommand{\rs}[1]{\textcolor{ForestGreen}{RM: #1}}
\newcommand{\rp}[1]{\textcolor{Magenta}{RP: #1}}
\newcommand{\pcm}[1]{\textcolor{Red}{PCM: #1}}
\newcommand{\jp}[1]{\textcolor{Blue}{JP: #1}}
\renewcommand{\baselinestretch}{1.0}
\setlength{\emergencystretch}{50pt}
\setcounter{secnumdepth}{2}
\renewcommand{\sparklineheight}{2} 
\renewcommand{\sparklinethickness}{0.5pt}
\renewcommand{\sparkspikewidth}{1.2pt} 
\definecolor{sparkspikecolor}{named}{cyan}
\definecolor{sparklinecolor}{named}{black}

\begin{document}

%\permission{Permission to make digital or hard copies of all or part of this work for personal or classroom use is granted without fee provided that copies are not made or distributed for profit or commercial advantage and that copies bear this notice and the full citation on the first page. To copy otherwise, to republish, to post on servers or to redistribute to lists, requires prior specific permission and/or a fee.}
%\conferenceinfo{WWW Montreal, Canada}{'16}
%\copyright{Copyright 2015 ACM X-XXXXX-XX-X/XX/XX ...\$15.00.}

%\title{Large-scale analysis of social media campaigns on climate change}
\title{Please Sign to Save ... : When Petitions on Environmental Causes Succeed}

%or \title{Sign and Save... : How Online Petitions Succeed}
% "Petitions on Environmental Causes" is not really a term according to Google


\newcommand{\etal}[1]{#1~\emph{et al.}}

%\numberofauthors{5} %  in this sample file, there are a *total*
% of EIGHT authors. SIX appear on the 'first-page' (for formatting
% reasons) and the remaining two appear in the \additionalauthors section.
%
%\author{
% You can go ahead and credit any number of authors here,
% e.g. one 'row of three' or two rows (consisting of one row of three
% and a second row of one, two or three).
%
% The command \alignauthor (no curly braces needed) should
% precede each author name, affiliation/snail-mail address and
% e-mail address. Additionally, tag each line of
% affiliation/address with \affaddr, and tag the
% e-mail address with \email.
%
 % 1st. author
% \alignauthor Julia Proskurnia \\
%        \affaddr{\'{E}cole Polytechnique F\'{e}d\'{e}rale de Lausanne}\\
%        \affaddr{Lausanne---Switzerland}\\
%        \email{iuliia.proskurnia@epfl.ch}
% 2nd. author
% \alignauthor Karl Aberer\\
% 	     \affaddr{\'{E}cole Polytechnique F\'{e}d\'{e}rale de Lausanne}\\
%        \affaddr{Lausanne---Switzerland}\\
%        \email{karl.aberer@epfl.ch}
% 3rd. author
% \alignauthor Philippe Cudr\'{e}-Mauroux\\
% 	    \affaddr{University of Fribourg}\\
%        \affaddr{Fribourg---Switzerland}\\
%        \email{phil@exascale.info}

\maketitle


\begin{abstract}

% Climate change
Social media have become one of the key platforms to support the debate on climate change.
% Campaigns and e-petitioning
In particular, Twitter allows easy information dissemination when running environmental campaigns.
Yet, the dynamics of these campaigns on social platforms still remain largely unexplored.
% Role of the petitions in the environmental campaigns
In this paper, we study the success factors enabling online petitions to attain their required number of signatures.
% Petition success based on the campaign or Twitter activity
We present an analysis of e-petitions and identify how their number of users, tweets and retweets correlate with their success. In addition, we discuss how campaigns with high user engagement are the most active in promoting petitions on Twitter \pcm{I don't understand that}.
% Findings...
Finally, we present an annotated corpus of petitions posted by environmental campaigns together with their corresponding tweets \pcm{why? towards what goal? seems disconnected to the res}.

\end{abstract}

 % A category with the (minimum) three required fields
% \category{H.3.1}{Information Storage and Retrieval}{Content Analysis and Indexing}
 %A category including the fourth, optional field follows...
% \category{J.4}{Social and Behavioral Sciences}{Sociology}
% \terms{Algorithms, Experimentation}
%\keywords{public campaign analysis, action extraction, time series analysis}

\vspace{1em}

\setlength{\parskip}{0pt} % 1ex plus 0.5ex minus 0.2ex}
\setlength{\parindent}{0pt}

%!TEX root = ../sig-alternate.tex
\section{Introduction}
\label{sec:intro}

% Context: 
% Environment and wildlife issues are linked to the human-induced climate change~\cite{solomon2009irreversible} and therefore, humanity is concerned about the ways to improve the situation.
%
% As a result, public campaigns have became one of the major instruments to increase awareness and mobilize people towards issues of climate change and animal welfare~\cite{Pearce2014}.
%
% In turn, public campaigns often leverage e-petitioning~\cite{mosca2009petitioning} to reach out to local, national governments or institutions.
% Outcome of the petitions is relatively easy to extract, since such information is usually provided by petition aggregator web cites. This information can help quantify the performance of the public campaigns and petitions themselves, i.e., by a successful petition we understand a petition that has reached specified number of signatures.

The discourse on climate change is often focused on the impact it has on the environment and on  wildlife~\cite{solomon2009irreversible}.
To bring these issues in the public spotlight, social media campaigns have proved to be an effective instrument to raise awareness and mobilize masses~\cite{Pearce2014}.
To further push for concrete actions from governments or public entities, many campaigns resort to e-petitioning~\cite{mosca2009petitioning}, whose success is also much easier to assess: reaching or not a required number of signatures.
Information about the number of signatures obtained for a given e-petition is often publicly available via e-petitions aggregators websites such as \url{thepetitionsite.com}, \url{avaaz.org}, \url{change.org} etc., and can be used as a proxy for the performance of the public campaigns and petitions themselves.

% Existing opportunity
%
% 1.
% E-petitions share many features and qualities with other social platforms, e.g., can create and share content.
% Since Twitter remains one of the easiest way to disseminate information on the Web to a large groups of people, we analyse tweets mentioning petitions posted with the environmental campaign hashtags on Twitter.
% It is important for the campaign leaders to know how petitions are promoted by other campaigns on Twitter and if a petition success is caused by any of the Twitter platform aspects (users that tweet the petitions, petitions tweets outside of the campaign, type of campaign, final goal of the petitions etc.)

% Our goal
%   Research questions and major findings:
In this work, we tackle two main research questions.

\textbf{RQ1:} \textit{What is the role of petitions in various types of environmental campaigns?}
To answer this question, we study several environmental campaigns than were run in the beginning of 2015, measuring the incidence of e-petitioning as an instrument for campaigning across different types of campaigns (awareness, mobilization). We find that petitioning is particularly important during mobilization campaigns.\footnote{Mobilization campaigns refer to the campaigns whose primary goal is to engage and motivate a wide range of partners, allies and individual at the national and local levels, towards a particular problem or issue. While awareness campaigns refer to the campaigns whose primary goal is to raise people’s awareness regarding a particular subject, issue, or situation.}

\textbf{RQ2:} \textit{What makes a petition promoted by a public campaign successful?} We answer this question by making a feature analysis and comparing tweets that belong to public campaigns to individual tweets. 
%
We propose a set of social and contextual features and show how the required number of signatures for an environmental petition is  correlated to its outcome.
Additionally, we release an annotated corpus with the petitions, their tweets and outcomes\footnote{Link is removed to ensure anonymity.}.
For this study we focus on Twitter, which remains one of the main channels for social media campaigns, also providing relatively easy access to campaign data.

% Related work and our position/contributions x 2 paragraphs
% Campaigns/Protests on Climate Change
\textbf{Climate Change Discourse on Social Media.} Climate change is a highly discussed topic. \cite{Kirilenko2014} overviews the climate change domain, its polarization, discussion over time etc. \cite{Olteanu2015} studies how various climate related events are highlights by various media sources.
Variety of public campaigns use social platforms to increase awareness or mobilize people~\cite{Mahmud2014}.
\cite{Tufekci2013} describes how the on-line attention can be driven towards particular politicized persona, while \cite{gonzalez2013networked} analyses information transmission during protests.
\cite{hestres2013preaching} studies public mobilization and online-to-offline social movement strategies for two major environmental movements. Unlike the prior work, we analyze a over 100 environmental campaigns as well as their effect on the petition success.
% TODO: Possible add to the previous section one of the findings we will have!!!

% Petitions
\textbf{Characterizing E-petitions.} Various works were conducted to analysis the e-petitions on various petition aggregators.
\cite{Hale2013} describes a temporal analysis of 8K petition on the UK No. 10 Downing Street and make an observation towards early signs of successful petition (large number of signatures during the first days).
\cite{Huang2015} analyses ``power'' users that produce petitions. The authors have shown that only 1\% of general topic petitions on \url{change.org} reaches their goal.
However, to the best of our knowledge, we are the first to analyze which factors predict the success of an environmental petition based on the internal and external attributes of the corresponding public campaign on Twitter.
% Kickstarter
On the other hand, analysis of the e-petitions can be compared to the crowdfunding, since in both fields desired and obtained support can be analysed. \cite{Etter2013} studies various prediction techniques of the Kickstarter campaigns.
Later, \cite{An2014} analyses investor activity on Kickstarter and make recommendations of projects based on their activity on Twitter. Unlike aforementioned works, we focus on the climate change and animal welfare petitions, as a part of the environmental public campaigns on Twitter.


% Implications/Impact etc...
In this work, we found that 25\% of the petition posted with environmental campaigns hashtags on Twitter obtained their required number of signatures.
Moreover, we identify a number of features that can act as indicators for the success of the petitions.
This information might be of a great interest to the environmental activists and campaign leaders as it can influence the success of the message they are conveying to the public.
It should be noted that the techniques presented below are not restricted to the environmental domain and could be applied to any related setting.

% Disclaimers
% In Section~\ref{sec:dataset}, we describe our dataset, its main characteristics, filtering and annotation we apply. We then present in Section~\ref{sec:petition_analysis} petitions' position within environmental campaigns and outside of then within Twitter platform. Finally, we highlight major finding (Section~\ref{sec:discussion}) and conclude in the end.
%!TEX root = ../sig-alternate.tex
\section{Data Collection, Cleansing and Insights}
\label{sec:dataset}

Our study relies on the collection of roughly 7500 tweets and retweets about a set of 240 petition related to the campaigns on climate change and animal welfare, which were posted from January 2015 to April 2015. Specifically, we consider a tweet to be related to a given petition if it contains a word ``petition'' in its content.

\textbf{Campaigns dataset and petition tweets:}
% * Environmental campaigns on Twitter and their collection by means of power users and crowdsourcing.
In order to analyze the role of the petitions in success of public climate change and animal welfare campaigns
we have obtained an annotated corpus of such hashtags\footnote{Link is removed to ensure anonymity.}.
The campaign corpus consists of 101 public environmental campaigns with over 850K unique tweets that spans over Jan 2015 - April 2015. 
We assume that each campaign has a uniquely identified hashtag, e.g., \#saveafricananimals, \#tweet4dolphins etc.
Moreover, all the campaign hashtags are labeled by (a) a high-level goal, e.g., awareness or mobilization type, and (b) a user engagement pattern over time, e.g., one-day campaigns, ever-growing, annual, inactive\footnote{One-day campaigns have the most user activity fallen into the start/first mention of the hashtag.
Ever-growing campaigns have constantly growing number of users posting with the hashtag.
Annual campaigns are mentioned annually, e.g., yearly, monthly. Inactive campaigns have very low user engagement overall.}.
Those are the main categories that will be used further in the analysis of the petition positioning across various types of campaigns. It should be noted that ``ever-growing'' campaigns are the most interesting ones since they maintain constantly growing number of people that are involved in their action on Twitter.

% * Elimination of duplicates and merge unobvious duplicates
We have extracted all ``petition'' tweets from the annotated collection of environmental public campaigns tweets.
Here we present an example of a tweet with a petition URL: \textit{``.@thetimes Petiton: Call for Safer Storage of Nuclear Waste in over 80 USA cities. http://tiny.cc/okzicx  \#SaveFukuChildren''}.
Such tweets were identified in 39 (out of 101) campaigns with a total tweets count of 15K out of which belonged to unique unresolved links (excluding tweets with broken links).
Further, we have resolved, stored and annotated all petition URLs. As a result we have found 294 unique petition links and 158 broken or outdated links.
For valid petition links we have also stored their resolved URL. We have further used this information to eliminate URLs that point to the same petition.
This process has resulted in 240 unique petitions.

% * Limitation for further exploration                      
\textbf{Tweets with petition:}
Regarding the question of what makes a petition successful as part of the public campaign,
it should be noted that the campaign tweets collection does not account for the overall distribution of the petition tweets across the whole Twitter. 
Therefore, we collected additional data as we describe below.
To minimize the bias in our collection, we further collect tweets that contain one of 240 petition via \url{backtweets.com}. For this task we have used the collection of the extracted URLs with their resolved links (if applicable) and requested \url{backtweets.com} to return all historical tweets that mention given URL.
Clearly, this still results in only a subset of the petition tweets since it does not account for the URL redirects and shortening. However, we aim for a best effort collection which gives us a clearer picture on the distribution of the petitions' tweets.
As a result, we have enriched the tweet collection with over 1700 new tweets without campaign hashtag.

\textbf{Thepetitionsite.com.} To compare campaign petitions with other environmental petitions, we have additionally collected all the environmental and animal welfare petitions from the major petition aggregator\footnote{Accessed on the 16th Feb 2016} \url{thepetitionsite.com} and corresponding tweets from \url{backtweets.com}. This resulted in over 2800 petition (79 of which were found in campaign dataset), where only 186 petitions were mentioned on Twitter with their direct URLs.

\textbf{Dataset preprocessing}
To be able to compare petitions with each other, we use both campaign and non-campaign tweets.
% TODO: Next sentence might be redundant...
A petition $p$ is characterized by its signature goal $S(p)$, collected signatures $C(p)$, $SignatureRate = \frac{C(p)}{S(p)}$ and the following set of Twitter related features $T_i(p)$.
\begin{enumerate}
	\setlength\itemsep{0em}
	\item Number of unique users posted the petition url;
	\item Number of tweets with url;
	\item Number of followers of the users posting petition tweets;
	\item Number of tweets with campaign hashtags vs without;
	\item Number of users that tweet a petition without campaign hashtag etc.
\end{enumerate}
%!TEX root = ../sig-alternate.tex
\section{Petition analysis}
\label{sec:petition_analysis}

Given a list of petition that gained tracking within climate change and animal welfare campaign on Twitter (described above), we, first, present an analysis on the petition position within different types of public campaigns and, second, analyse petition spread outside of the campaign. Finally, we propose a model to predict success of a particular campaign petition on Twitter.

\subsection{Petition in public campaigns on Twitter}
% Information regarding environmental campaigns, types etc.
% Our findings:
% (a) only mobilization and single awareness; annual campaign with big number of msgs and failure rate.
% (b) doimated by the ever-growing campaigns.
% (c) Interesting examples (http://www.petitions.moveon.org/sign/legalize-hemp-farming/ - askdrh;
% https://you.38degrees.org.uk/petitions/stop-the-fracking-cover-up-by-defra - talkfracking (fracking censorship to delete.))

Out of the 118 environmental campaigns in our dataset, only 20 campaign hashtags co-occur with a ``petition'' keyword.
Interestingly, the petitions were mainly initiated and promoted by mobilization campaigns, while only two awareness campaigns published petition with their hashtags, e.g., ``\#talkfracking'', ``\#worldlovefordolphins''. Interestingly, awareness petition were not directed towards a particular action but rather long term optimisations, e.g., preventing ``covering up'' hydraulic fracturing by some organizations, or legalize hemp farming.

From the perspective of user engagement pattern for the campaigns mentioning petitions over 60\% of them were identified as ever-growing, while the rest were distributed between ``annual'' and ``inactive'' campaigns. Not surprisingly, ``one-day'' campaigns do not tend to use petitions as their instruments since they use more short-term leverage. On the other hand, ``annual'' campaigns tend to have a large number of different petitions thus the fail rate is higher.

Surprisingly, while Kickstarter\footnote{kickstarter.com} campaigns~\cite{Etter2013} tend to fail for high final goals, while completely opposite situation can be observed for, usually non-profitable, environmental campaigns.

\subsection{Environmental petition on Twitter}
\jp{\lipsum[1-9]}

\subsection{Data analysis}
\label{sec:discussion}
\jp{\lipsum[1-2]}
%!TEX root = ../sig-alternate.tex
\section{Conclusions}
% Conclusions and main results.
In this paper we introduce a dataset of environmental petitions that were promoted by major environmental public campaigns on Twitter.
We study the petition role as one of the actions performed by a public campaign.
We propose a model to identify successful petitions by their presence on Twitter and highlight the main aspects featuring a petition to obtain required number of signatures.
% We show that the more critical petition is, the more likely it is to have a high signature goal and the more likely it is to collect signatures.
Although, our dataset is limited in size, we made a best effort data collection and cleaning, and could observe the petition spread within the public environmental campaigns and identify the major factors that may lead to the success of the petition.
Our findings can provide helpful directions for all leaders of the public campaigns, its participants, petition initiators and signers.

% Future work and limitation.
% 1.
We plan to enhance the petition dataset by repeating the collection process over larger span of time from Twitter.
Another interesting future direction is to study the user aspect of the petition promoters on Twitter. In particular, we could identify the relations between petition signers and users who promote petitions on Twitter. The main difficulty here is to obtain this information for the large number of petition.

% 2.
Currently we propose a set of basic features and highlight the most valuable ones. The next step would be to explore time series properties of signatures, as well as, give actionable feedback on how to increase number of signers.

\bibliographystyle{aaai}
{
\begin{small}
\balance
\bibliography{sigproc}
\end{small}
}

\end{document}
